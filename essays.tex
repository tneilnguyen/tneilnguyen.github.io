\documentclass[oneside, a5paper]{book}
\usepackage{fontspec,geometry}
\setmainfont{Scala Pro}
\begin{document}
\title{Essays for Bingo}
\author{Tyler Nguyen}
\maketitle
\chapter{I Hate it Here}
Anyone who has had the dubious pleasure of my company over the past year has likely endured an unsolicited polemic against suburban sprawl; inadvertently and by choice, I have come to more fully appreciate suburbia, presently dominated by vast tracts of single-family homes, as a stubbornly central feature of life in the United States. Here, I hope to recount my steps toward this more complete understanding and explore briefly what, in my mind, are the broad contours of the suburban experience.

My journey into the riveting world of city planning discourse began when a video from \textit{Not Just Bikes} surfaced, rather innocuously, on my YouTube homepage. Sensing from the unornamented thumbnail and title, which indicated its position in a lengthy series, that the video could be painfully dry, I ignored it; however, it remained persistently among my suggestions, and after assurances from a friend that it was worth watching, I relented. Driven by an intense and almost morbid thrill of recognition, I watched dozens of the channel's videos, developing a previously latent resentment for the geographically expansive yet paradoxically circumscribed lifestyle enforced by the structure of the cities examined in each episode. Having experienced firsthand this dreary reality in several locations across North America, the presenter had moved to Amsterdam, an exemplar of a more deliberate, considered form of city planning and a convenient foil for the more diffuse, ad-hoc localities he'd left. Where \textit{Not Just Bikes} succeeds is in persuading viewers that cities configured around pedestrian, cyclist, and train access are a more than compelling alternative to ones which require automobiles to traverse expediently; through focused comparisons of life's day-to-day tenor (running errands, commuting to work, engaging in recreation) within each paradigm, the presenter assembles a comprehensive case for the superior quality of life, accessibility, health, safety, and even financial strength of communities which emphasize scalable access to more diverse forms of transit and accord privilege to people, not their cars. I fear, however, that \textit{Not Just Bikes} ultimately misapprehends the crux of our predicament; the implicit and explicit exhortations to appeal to municipal leaders for incremental change towards more livable cities betray a central na\"ivet\'e, a belief that the defacement of the North American landscape was comprised of a series of fashionable mistakes by well-meaning, if myopic planners.

In \textit{Building Suburbia}, Dolores Hayden offers a more accurate, sobering assessment: the lucrative symbiosis of government regulators and bureaucrats, industrial-scale developers, and the real estate lobby has produced unchecked growth subsidized directly by insured mortgages, incentivized by federal tax law, mandated by zoning codes, and fed by an ever-expanding interstate highway system. As demonstrated there and with greater force and depth in \textit{The Color of Law}, policymakers saw the single-family home as a conduit to the middle class and a bulwark of its capitalist, anti-communist values, a cudgel with which to divide communities and render toothless the constitutional imperative of integration, and as a means of perpetuating a pernicious inequity through generational wealth. Recognizing a sea change in the objectives of housing policy is to confront the fundamental gap between perceived and real agency one has as a citizen and resident; no amount of pressure at the local level has any hope of enacting the transformational change needed to halt the cycle of unlimited development and spur reinvestment in denser, livable neighborhoods, and, despite often defensive remarks to the contrary, few can be said to have freely chosen to live this way.
\end{document}
