\documentclass[oneside, a5paper]{book}
\usepackage{fontspec,geometry}
\setmainfont{Nexus Serif Pro}
\begin{document}
\title{Essays for Bingo}
\author{Tyler Nguyen}
\maketitle
\chapter{I Hate it Here}
Anyone who has had the dubious pleasure of my company over the past year has likely endured an unsolicited polemic against suburban sprawl; inadvertently and by choice, I have come to more fully appreciate suburbia, presently characterized by vast tracts of single-family homes, as a stubbornly universal feature of life in the United States. Here, I hope to recount my steps toward this more complete understanding and explore briefly what, in my mind, are the broad contours of the suburban experience.

My journey into the riveting world of city planning discourse began when a video from \textit{Not Just Bikes} surfaced, rather innocuously, on my YouTube homepage. Sensing from its unornamented thumbnail and title, which revealed its place within a lengthy series, that the video could be intolerably dry, I ignored it; however, it remained persistently among my suggestions, and after assurances from a friend that appearances were deceiving, I relented. Driven by an intense and almost morbid thrill of recognition, I imbibed dozens of the channel's videos, developing a previously latent resentment for the geographically expansive yet paradoxically circumscribed lifestyle enforced by the structure of the cities examined in each episode. Having experienced firsthand this dreary reality in several locations across North America, the presenter had moved to Amsterdam, an exemplar of a more deliberate, considered form of city planning and a convenient foil for the diffuse, ad-hoc localities he'd left. Where \textit{Not Just Bikes} succeeds is in persuading viewers that cities configured around pedestrian, cyclist, and train access are a more than compelling alternative to ones which require automobiles to traverse expediently; through focused comparisons of life's tenor (running errands, commuting to work, engaging in recreation) within each paradigm, the presenter assembles a comprehensive case for the superior accessibility, health, safety, and even financial strength of communities which emphasize scalable access to more diverse forms of transit and accord privilege to people, not their cars. I fear, however, that \textit{Not Just Bikes} ultimately misapprehends the crux of our predicament; the implicit and explicit exhortations to appeal to municipal leaders for incremental change towards more livable cities betray a central na\"ivet\'e, a belief that the defacement of the North American landscape was comprised of a series of once-fashionable mistakes by well-meaning, if myopic planners.

In \textit{Building Suburbia}, Dolores Hayden offers a more accurate, sobering assessment: the lucrative symbiosis of government regulators and bureaucrats, industrial-scale developers, and the real estate lobby has produced unchecked growth subsidized directly by insured mortgages, incentivized by federal tax law, mandated by zoning codes, and fed by an ever-expanding interstate highway system; municipalities bear the fiscal and social consequences. As demonstrated there and with greater force and depth in \textit{The Color of Law}, policymakers saw the single-family home as a conduit to the middle class and a bulwark of its capitalist, anti-communist values, a cudgel with which to divide communities and render toothless the constitutional imperative of integration, and as a means of perpetuating corrosive inequity through generational wealth. Recognizing that stemming the tide requires a sea change in the basic objectives of housing policy is to confront the fundamental gap between perceived and real agency one has as a citizen and resident; no amount of pressure at the local level has any hope of enacting the transformational change needed to halt the cycle of unlimited development and spur reinvestment in denser, livable neighborhoods, and, despite often defensive remarks to the contrary, few can be said to have freely chosen to live this way, absent exposure to any alternative.

How, then, does it feel to inhabit these spaces? Susan Faludi, in \textit{Stiffed}, outlines on the gendered expectations written into the post-war suburban experiment: for a housewife, the tract home is a sort of gilded confinement, one in which she has relinquished her independence for the peripheral role of chief consumer, while for a husband, it is a dominion, his to command and provision. It is the former that Faludi identifies as emblematic of the period's nascent advertising-saturated culture; the preponderance of the consumer has not waned since. I think this drives at the core of that vague sense of alienation that suffuses the suburban experience; as a child, one is chauffeured, by necessity, between academic, extracurricular, and commercial spaces. The latter becomes most prominent as consumption, observed from a distance, becomes aspirational; participation in the market becomes the mark of maturity and personhood. Children may, of course, attempt to superimpose fun onto an environment which, by its nature, is incompatible with organic play; grand boulevards elevate motorists, while the dearth of accessible shared spaces indicate that everyone else is merely tolerated. The dead air of childhood ennui, an inevitable consequence of the time spent in transit and the curtailed freedom resulting from being tethered to a parent, can now be transformed into yet more ad space as children occupy themselves on the Internet. Neighbors begin to regard the very idea of community with cynicism as they observe each other over their fences, held in mutual suspicion. Adults are left adrift in a sea of asphalt, dotted with strip malls, supermarkets, and shopping centers; from within indistiguishable outposts of mass retailers or franchise restaurants, devoid of any sense of place, escape seems impossible.

\section*{Musical accompaniment}
\begin{itemize}
    \item Food House -- ``ride'': ``Parking lot, parking lot / Party in the parking lot''
    \item Big Black -- ``Kerosene'': ``Nothing to do, sit around at home / Sit at home, stare at the walls / Stare at each other and wait till we die''
\end{itemize}
\end{document}
